\documentclass[a4paper,11pt]{article}

\usepackage{mathptmx}% Times Roman font
\usepackage{CJKutf8}  %支持中文
\usepackage[T1]{fontenc}
\usepackage[utf8]{inputenc}

\usepackage[unicode={true}]{hyperref} 

\usepackage{graphicx}   %表格
\usepackage{booktabs,longtable,lscape}
\usepackage{pdflscape}
\usepackage{xcolor}
\usepackage{caption}
\usepackage{colortbl}   %表格底色
\usepackage{float}      %固定浮动窗口

\usepackage{graphicx} %package to manage images
\graphicspath{ {png/} }
\usepackage{subfigure} %双栏图片
\usepackage{footbib}  %参考文献

\usepackage{tgtermes}
\usepackage{amsmath,amssymb,amsthm,textcomp}
\usepackage{enumerate}
\usepackage{multicol}
\usepackage{tikz}

\usepackage{geometry}
\geometry{total={210mm,297mm}, left=25mm, right=25mm, bindingoffset=0mm, top=20mm,bottom=20mm}


\linespread{1.3}

\newcommand{\linia}{\rule{\linewidth}{0.5pt}}

% table
\makeatletter
\newcommand{\figcaption}{\def\@captype{figure}\caption}
\newcommand{\tabcaption}{\def\@captype{table}\caption}
\makeatother

\newcommand{\tabincell}[2]{\begin{tabular}{@{}#1@{}}#2\end{tabular}}
%然后使用&\tabincell{c}{}&就可以在表格中自动换行

\renewcommand\arraystretch{0.9}   %行距

% custom theorems if needed
\newtheoremstyle{mytheor}
{1ex}{1ex}{\normalfont}{0pt}{\scshape}{.}{1ex}
{{\thmname{#1 }}{\thmnumber{#2}}{\thmnote{ (#3)}}}

\theoremstyle{mytheor}
\newtheorem{defi}{Definition}

% my own titles
\makeatletter
\renewcommand{\maketitle}{
	\begin{center}
	\vspace{2ex}
	{\huge \textsc{\@title}}
	\vspace{1ex}
	\\
		\linia\\
		\@author \hfill \@date
		\vspace{4ex}
	\end{center}
}
\makeatother

% custom footers and headers
\usepackage{fancyhdr,lastpage}
\pagestyle{fancy}
\lhead{}
\chead{}
\rhead{}
\lfoot{v1.0}                  %版本号
\cfoot{}
\rfoot{ \thepage\ /\ \pageref*{LastPage}}
\renewcommand{\headrulewidth}{0pt}
\renewcommand{\footrulewidth}{0pt}
%

%%%----------%%%----------%%%----------%%%----------%%%

\begin{document}
\begin{CJK*}{UTF8}{gbsn}

\title{新生儿V1丹麦测试数据评估报告}
\author{LAN, Zhangzhang}
\maketitle

\section{下机数据}
\begin{itemize}\setlength{\itemsep}{0pt}
\item[-] 下机时间:20150817
\item[-] 下机路径:/share/Fastq02/solexa/F14ZQSQSET1523\_HUMpmkX/DKDNAPEP01529 (丹麦)
\item[-] 实验目的:重复样本在丹麦实验室建库以及测序,深圳集群分析,验证结果一致性。
\item[-] 实验设计: 常规建库。本次测试共10个样本,2个YH为重复样本,8个阳性样本。以往数据均为酶切样本,不能作为互相比较。实验方案请参考《新生儿hiseq平台验证实验方案.docx》。
\item[-] 芯片特性:华大自制芯片,目标区域包括218个基因的外显子及外显子前后50bp,UTR和深度内含子的区域,大小共1,004,680bp。其中线粒体探针捕获大部分数据量。
\end{itemize}


\begin{center}
\fontsize{10}{12} \selectfont
\begin{tabular}[t]{l|l|l|l|l|l}
\hline
序号	&	样本名称	&	性别	&	 文库号	&	样品类型	&	使用量	\\
\hline
1	&	SD13S0056	&	M	&	pre-pcr-SD13S0056-43	&	干血片	&	200ng	\\
2	&	SD12S0040	&	M	&	pre-pcr-SD12S0040-28	&	干血片	&	200ng	\\
3	&	SD13S0539	&	M	&	pre-pcr-SD13S0539-39	&	干血片	&	200ng	\\
4	&	13B0005773	&	M	&	pre-pcr-13B0005773-44	&	干血片	&	200ng	\\
5	&	SD13S0244	&	F	&	pre-pcr-SD13S0244-41	&	干血片	&	200ng	\\
6	&	SD13S0213	&	F	&	pre-pcr-SD13S0213-42	&	干血片	&	200ng	\\
7	&	SGDV1P297	&	M	&	pre-pcr-SGDV1P297-25	&	干血片	&	200ng	\\
8	&	13B0005808	&	M	&	pre-pcr-13B0005808-31	&	干血片	&	200ng	\\
9	&	YH-2	&	M	&	pre-pcr-HYDNA-46	&	干血片	&	200ng	\\
10	&	YH-1	&	M	&	pre-pcr-HYDNA-45	&	干血片	&	200ng	\\
\hline
\end{tabular}
\tabcaption{测试样本列表}
\end{center}

\section{比对结果统计分析}
\subsection{深度与覆盖度}
\input{coverage_report}
\fontsize{10}{12} \selectfont
关于表中内容的说明:
\begin{itemize}\setlength{\itemsep}{0pt}
\item[-] PCR duplication:20\%-30\%为正常值。两个炎黄样本却异常偏高,为60\%。
\item[-] 样品捕获效率很高,但样本之间略微差异。深圳酶切样本最高为40\%。
\item[-] 样本的平均深度达到400×以上,新生儿产品的深度要求达到150×以上,可稍微减少数据量,以节约成本。
\item[-] 样本覆盖度能达到99\%,符合要求。
\item[-] 样本区域均一性较好。
\end{itemize}

\subsection{各染色体捕获数据量百分比}
\input{chromosome_report}
\fontsize{10}{12} \selectfont
\begin{itemize}\setlength{\itemsep}{0pt} 
\item[-] 捕获线粒体数据量远远高于核基因组数据量,与芯片预期一致。
\end{itemize}

\subsection{核基因深度}
\begin{itemize}\setlength{\itemsep}{0pt}
\item[-] 由于该芯片特性,线粒体捕获了大部分数据量。下面数据结果为去掉线粒体区域,单独分析核基因区域。
\item[-] 深度,捕获效率等都是符合生产要求的.
\item[-] 具体结果如下。
\end{itemize}
\begin{center}
\renewcommand{\arraystretch}{1}
\fontsize{8}{12} \selectfont
\begin{tabular}[t]{l|l|l|l|l|l}
\hline
title	&	13B0005808	&	SD13S0056	&	13B0005773	&	SD13S0213	&	SD13S0539	\\
\hline
Average depth	&	310.61	&	278.51	&	294.87	&	332.95	&	349.78	\\
Average depth(rmdup)	&	258.02	&	233.81	&	243.86	&	275.79	&	288.3	\\
Fraction of Target Reads in all reads	&	31.86\%	&	37.90\%	&	36.16\%	&	34.61\%	&	37.05\%	\\
Fraction of Target Reads in mapped reads	&	31.87\%	&	37.91\%	&	36.17\%	&	34.62\%	&	37.06\%	\\
Fraction of Target Data in all data	&	28.42\%	&	33.77\%	&	32.22\%	&	30.88\%	&	33.10\%	\\
Fraction of Target Data in mapped data	&	28.42\%	&	33.78\%	&	32.23\%	&	30.89\%	&	33.11\%	\\
\hline
title	&	SD12S0040	&	SGDV1P297	&	SD13S0244	&	YH-1	&	YH-2	\\
\hline
Average depth	&	271.83	&	393.1	&	307.9	&	414.05	&	401.5	\\
Average depth(rmdup)	&	225.25	&	319.57	&	256.85	&	336.8	&	327.3	\\
Fraction of Target Reads in all reads	&	44.12\%	&	39.06\%	&	31.22\%	&	12.40\%	&	11.93\%	\\
Fraction of Target Reads in mapped reads	&	44.13\%	&	39.07\%	&	31.22\%	&	12.40\%	&	11.94\%	\\
Fraction of Target Data in all data	&	39.35\%	&	34.89\%	&	27.88\%	&	11.08\%	&	10.66\%	\\
Fraction of Target Data in mapped data	&	39.36\%	&	34.90\%	&	27.88\%	&	11.08\%	&	10.66\%	\\
\hline
\end{tabular}
\tabcaption{各样本核基因深度列表}
\end{center}

\section{variation准确性及一致性分析}
\subsection{阳性位点检出情况}
8个样本的阳性位点,包括SNP和indel,数据均与深圳测试结果一致。\\
\renewcommand{\arraystretch}{1}
\fontsize{8}{12} \selectfont	
\begin{tabular}[t]{l|l|l|l|l|l}
\hline
序号	&	样本名称	&	性别	&	 文库号	&	阳性位点	&	检出情况	\\
\hline
1	&	SD13S0056	&	M	&	pre-pcr-SD13S0056-43	&	GJB2:c.235delC(Hom)	&	检出	\\
2	&	SD12S0040	&	M	&	pre-pcr-SD12S0040-28	&	BCKDHB:c.580C>T(Het) &	检出	\\
3	&	SD13S0539	&	M	&	pre-pcr-SD13S0539-39	&	GJB2:c.235delC(Hom)	&	检出	\\
4	&	13B0005773	&	M	&	pre-pcr-13B0005773-44	&	PAH:c.331C>T(Het);c.728G>A(Het)	&	检出	\\
5	&	SD13S0244	&	F	&	pre-pcr-SD13S0244-41	&	\tabincell{c}{SLC25A13:c.852\_855delTATG(Het);\\ c.1177+1G>A(Het)} & 检出	\\
6	&	SD13S0213	&	F	&	pre-pcr-SD13S0213-42	&	CYP21A2:c.955C>T(Het)	&	未检出(深圳也未检出)	\\
7	&	SGDV1P297	&	M	&	pre-pcr-SGDV1P297-25	&	ASL:p.Asp145Gly(het)	&	检出	\\
8	&	13B0005808	&	M	&	pre-pcr-13B0005808-31	&	TSC2:c.5048T>G(het)	&	检出	\\
9	&	YH-2	&	M	&	pre-pcr-HYDNA-46	&	无	&	无	\\
10	&	YH-1	&	M	&	pre-pcr-HYDNA-45	&	无	&	无	\\
\hline
\end{tabular}
\tabcaption{测试样本列表}

\subsection{YH一致性分析}
NGS检测方法的准确性可以通过分析SNP假阳性和假阴性的个数来进行评估。灵敏度是指检测到目标区域中突变点的可能性,该指标反应的是检测的假阴性率。NGS特异性是指在目标区域中检测不到不存在突变点的概率,该指标反应检测的假阳性率。
首先,本次测试使用YH样品,以炎黄数据库为为参考标准,计算两个炎黄样本YH-1和YH-2 的SNP一致性,以此评估。其次,查看以往深圳酶切炎黄样本YH-daily,比较两次实验的一致性,以此评估。此次测试的两个炎黄样本表现优异,深圳以往样本则因为酶切建库的影响,结果稍差。
具体结果见下表, 计算方法见备注参考文献\footcite{Harismendy2009}。
\subsubsection{常规建库YH-1、YH-2一致性评估}
\begin{center}
\begin{tabular}[t]{l|l|l}
\hline
此次测试	&	YH-1	&	YH-2	\\
\hline
		False positive rate	&	0.00000	&	0.00000	 \\
		False negative rate	&	0.00000	&	0.00000	 \\
		Variant discrepancy rate & 0.02116 & 0.02116 \\ 
		Coverage rate 	&	0.82968 	&	0.82968	\\
\hline
\end{tabular}
\end{center}
\subsubsection{深圳酶切炎黄样本YH-daily}
\begin{center}
\begin{tabular}[t]{l|l}
\hline
以往样本	&	YH-daily	\\
\hline
		False positive rate	&	0.00000		 \\
		False negative rate	&	0.00000		 \\
		Variant discrepancy rate & 0.04712  \\ 
		Coverage rate 	&	0.83097	\\
\hline
\end{tabular}
\end{center}
\footbibliography{mybib}
\footbibliographystyle{abbrv}

\section{附录}
\fontsize{10}{12} \selectfont
\begin{itemize}\setlength{\itemsep}{0pt}
\item[-] 限于篇幅,只列出部分信息。
\item[-] 余下信息如需要可邮件lanzhangzhang@。
\item[-] 由于线粒体占据了大部分数据量,所以在以下的图表中,均是以核基因区域为目标区域而计算的。
\item[-] 外显子深度分布图则是选择YH-1,随机几个基因绘制。结果显示均一性较好。
\end{itemize}
\subsection{单碱基分布图}
\include{qc}

\subsection{外显子深度分布图}
\input{graph}


\end{CJK*}
\end{document}
